\documentclass[12pt]{book}
\usepackage[utf8]{inputenc}
\usepackage[utf8]{inputenc}
\usepackage[T1]{fontenc}
\usepackage{pdfsync}
\usepackage[brazilian]{babel}
\usepackage{amssymb}
\usepackage{mathpartir}
\usepackage{color}
\usepackage{amsmath}
\usepackage{amssymb}
\usepackage{amsthm}
\usepackage{enumerate}
\usepackage{ulem}
\usepackage{xspace}
\usepackage[boxed,lined]{algorithm2e}
\usepackage{hyperref}
%\usepackage{courier}
\usepackage{ulem}
\usepackage{makeidx}
\makeindex

\usepackage{fixltx2e}
\usepackage{graphicx}
\usepackage{longtable}
\usepackage{float}
\usepackage{wrapfig}
\usepackage{rotating}
\usepackage{textcomp}
\usepackage{marvosym}
\usepackage{wasysym}
\usepackage{fancyhdr} %For headers and footers

\renewcommand{\baselinestretch}{1.4}

\usepackage{xcolor}
\definecolor{myGreen}{RGB}{0,100,0}
\definecolor{myRed}{RGB}{230,0,0}
\definecolor{red}{RGB}{230,0,0}
\definecolor{blue}{RGB}{0,0,205}
\definecolor{myRed2}{RGB}{210,0,0}
\definecolor{myBlue}{RGB}{0,0,205}
\usepackage{setspace}
\usepackage{bm}
\usepackage{graphicx}
\bibliographystyle{plain}
\setlength{\parindent}{0.5cm}
\usepackage{proof}
%\usepackage{tikz}
\input xy
\xyoption{all}

\usepackage[margin=2.5cm,top=2cm,left=2.5cm]{geometry}
\usepackage{multicol}

\addto\captionsenglish{\renewcommand{\chaptername}{Capítulo}}

\newtheorem{definicao}{Definição}
\newtheorem{teorema}{Teorema}
\newtheorem{lema}{Lema}
\newtheorem{corolario}{Corolário}
\newtheorem{observacao}{Observação}
\newtheorem{nota}{Nota}
\newtheorem{notacao}{Notação}
\newtheorem{exemplo}{Exemplo}
\newtheorem{exercicio}{Exercício}

\long\def\ignore#1{\relax}


\title{Projeto e Análise de Algoritmos - Notas colaborativas}
\author{Flávio L. C. de Moura}
\begin{document}
\maketitle

\chapter{Fundamentos Matemáticos}

\chapter{Notação Assintótica}

\begin{definicao}[O conjunto $O(g)$] Seja $g$ uma função dos inteiros não-negativos nos reais positivos. Então $O(g)$ é o conjunto das funções (também dos inteiros não-negativos nos reais positivos) tal que existem uma constante real $c>0$ e uma constante inteira $n_0>0$ satisfazendo a desigualdade $f(n) \leq c\cdot g(n), \forall n \geq n_0$. Alternativamente, $O(g(n))$ = \{$f(n)$ : existem constantes positivas $c$ e $n_0$ tais que $0 < f(n) \leq c\cdot g(n), \forall n \geq n_0$. \}
\end{definicao}

\begin{definicao}[O conjunto $\Omega(g)$] Seja $g$ uma função dos inteiros não-negativos nos reais positivos. Então $\Omega(g)$ é o conjunto das funções (também dos inteiros não-negativos nos reais positivos) tal que existem uma constante real $c>0$ e uma constante inteira $n_0>0$ satisfazendo a desigualdade $c\cdot g(n) \leq f(n), \forall n \geq n_0$. Alternativamente, $\Omega(g(n))$ = \{$f(n)$ : existem constantes positivas $c$ e $n_0$ tais que $0 < c\cdot g(n)  \leq  f(n), \forall n \geq n_0$. \}
\end{definicao}

\begin{definicao}[O conjunto $\Theta(g)$] Seja $g$ uma função dos inteiros não-negativos nos reais positivos. Então $\Theta(g)$ é o conjunto das funções (também dos inteiros não-negativos nos reais positivos) tal que existem constantes reais positivas $c_1$ e $\cite{} _2$ e uma constante inteira $n_0>0$ satisfazendo a desigualdade $c_1\cdot g(n) \leq f(n) \leq c_2 \cdot g(n), \forall n \geq n_0$. Alternativamente, $\Theta(g(n))$ = \{$f(n)$ : existem constantes positivas $c_1, c_2$ e $n_0$ tais que $0 < c_1\cdot g(n)  \leq  f(n) \leq c_2 \cdot g(n), \forall n \geq n_0$. \}
\end{definicao}

\begin{lema}
  Uma função $f \in O(g)$ se $\displaystyle\lim_{n \to \infty}\frac{f(n)}{g(n)} = c < \infty$, incluindo o caso em que $c=0$.
\end{lema}

\begin{lema}
  Uma função $f \in \Omega(g)$ se $\displaystyle\lim_{n \to \infty}\frac{f(n)}{g(n)} > 0$, incluindo o caso em que o limite é igual a $\infty$.
\end{lema}

\begin{lema}
  Uma função $f \in \Theta(g)$ se $\displaystyle\lim_{n \to \infty}\frac{f(n)}{g(n)} = c$, para alguma constante $0< c < \infty$.
\end{lema}

\begin{definicao}[Complexidade do Pior Caso] Sejam $D_n$ o conjunto das entradas de tamanho $n$ para o algoritmo em questão, e $I\in D_n$. Seja $t(I)$ o número de operações básicas executadas pelo algoritmo na entrada $I$. Definimos a função $W$ por
$$W(n) = \max\{ t(I) \mid I \in D_n\}$$
\end{definicao}

\begin{definicao}[Complexidade do Caso Médio] Sejam $D_n$ o conjunto das entradas de tamanho $n$ para o algoritmo em questão, e $I\in D_n$. Sejam $t(I)$ o número de operações básicas executadas pelo algoritmo na entrada $I$, e $Pr(I)$ a probabilidade da entrada $I$ ocorrer. Definimos a função $A$ por
$$A(n) = \displaystyle\sum_{I\in D_n} Pr(I)\cdot t(I)$$
\end{definicao}

\begin{lema}
  Se $f \in O(g)$ e $g \in O(h)$ então $f \in O(h)$, ou seja $O$ é transitiva. Também são transitivos $\Omega, \Theta, o$ e $\omega$.
\end{lema}

\begin{lema}~ 
  \begin{enumerate} 
\item $f \in O(g)$ se, e somente se $g \in \Omega(f)$.
\item Se $f \in \Theta(g)$ então $g \in \Theta(f)$.
\item $\Theta$ define uma relação de equivalência sobre as funções. Cada conjunto $\Theta(f)$ é uma classe de equivalência que chamamos de \emph{classe de complexidade}.
\item $O(f+g) = O(\max\{f,g\})$. Equações análogas valem para $\Omega$ e $\Theta$. Estas equações são úteis na análise de algoritmos complexos onde $f$ e $g$ podem descrever o trabalho feito em diferentes partes do algoritmo.
\end{enumerate}
\end{lema}

\chapter{Algoritmos de Ordenação}

\section{Ordenação em Tempo Linear}

\chapter{Algoritmos em Grafos}

\begin{definicao}
  Um \textbf{grafo} (não dirigido) $G$ é um par $(V,E)$ onde $V$ é um conjunto finito não-vazio, e $E$ é um conjunto de pares não-ordenados de elementos de $V$. Um \textbf{digrafo} (ou um grafo dirigido) $G$ é um par $(V,E)$ onde $V$ é um conjunto finito não-vazio, e $E$ é uma relação binária sobre $V$.
  \end{definicao}

\end{document}

%%% Local Variables: 
%%% mode: latex
%%% TeX-master: t
%%% End: 

